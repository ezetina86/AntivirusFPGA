\chapter{Primer entorno generado}

\section{Introducción}

Una vez que se ha realizado la compilación del \emph{kernel} y se ha creado el
\emph{root file system} se cuenta con los recursos necesarios para realizar
pruebas sobre la tarjeta de desarrollo \emph{XUPV2P}.

\section{Formato de la \emph{compact flash}}

La \emph{compact flash}  deberá de tener  dos particiones para su correcto
funcionamiento en este proyecto, la primera partición será en formato FAT16 y en
donde estará localizado el archivo ``system.ace'', la segunda en formato
\emph{ext2} en donde se encontrará el \emph{root file system}.

La primera partición deberá de ser formateada bajo un sistema operativo
\emph{Windows} con la herramienta \emph{mkdosfs}\footnote{\emph{mkdosfs} es el
comando para crear un sistema MS-DOS FAT32 archivos bajo Linux, y como tal
similar en uso a mkfs.} con el siguiente comando:

\begin{verbatim}
 mkdosfs -s 64 -F 16 -R 1 F:
\end{verbatim}

Esta partición contendra el \emph{root file system} genrado anteriormente.

A la segunda partición se le asignará un formato \emph{ext2}, en esta partición
deberá de ser copiado el \emph{system.ace}.

\section{Modificación del \emph{device tree}}

Para poder juntar el \emph{kernel} con el \emph{root file system} generado será
necesario hacer modificaciones en el archivo ``xilinx.dts''  modificando la
siguiente linea  :

\begin{lstlisting}[caption=Archivo ``xilinx.dts'' ]
		bootargs = "console=ttyUL0,115200 root=/dev/ram, rw";
\end{lstlisting}
por :
\begin{lstlisting}[caption=``xilinx.dts'' ]
		bootargs = "console=ttyUL0,115200 root=/dev/xs`2 rw ip=dhcp";
\end{lstlisting}

Esta nueva configuración  permite al sistema montar el \emph{root file system}
desde la tarjeta \emph{compact flash}, además solicita a un servidor dhcp una
dirección IP. Debido a las modificaciones hechas al archivo ``xilinx.dts'' es
necesario recompilar el kernel y generar de nueva cuenta el
archivo ``system.ace''.


\section{Configuración de terminal}

Una vez que se han realizado las operaciones anteriores y se han copiado los 
archivos a la tarjeta \emph{compact flash} se proceden a realizar pruebas en
la tarjeta de desarrollo, para ello es necesario configurar una consola remota,
en este proyecto se hizo uso de \emph{Minicom}\footnote{Minicom es un programa
de módem basado en texto y emulación de terminal para sistemas operativos
\emph{UNIX}, originalmente escrito por Miquel Van Smoorenburg, y modelado de
acuerdo con el popular programa de MS-DOS Telix.}, la configuración se realiza
con los siguientes comandos:

\begin{verbatim}
  Proyecto@debian$sudo minicom -s
              +-----[configuration]------+
            | Filenames and paths      |
            | File transfer protocols  |
            | Serial port setup        |
            | Modem and dialing        |
            | Screen and keyboard      |
            | Save setup as dfl        |
            | Save setup as..          |
            | Exit                     |
            | Exit from Minicom        |
            +--------------------------+
    +----------------------------------------------------------+
    | A -    Serial Device      : /dev/USB0                    |
    | B - Lockfile Location     : /var/lock                    |
    | C -   Callin Program      :                              |
    | D -  Callout Program      :                              |
    | E -    Bps/Par/Bits       : 115200 8N1                   |
    | F - Hardware Flow Control : Yes                          |
    | G - Software Flow Control : No                           |
    |                                                          |
    |    Change which setting?                                 |
    +----------------------------------------------------------+
    
\end{verbatim}

\section{Arranque del sistema}

En este punto se tiene configurada la terminal para poder acceder a la tarjeta
de desarrollo, una vez que se coloca la \emph{compact flash} se enciende el
sistema el cual genera una salida mostrada en el el \emph{Apéndice B}.

Durante el arranque del sistema se generan llaves para la conexión via remota,
esto se puede observar de la siguente manera:

\begin{verbatim}
 Generating RSA Key...
Generating public/private rsa1 key pair.
Your identification has been saved in /etc/ssh_host_key.
Your public key has been saved in /etc/ssh_host_key.pub.
The key fingerprint is:
d7:2c:95:b1:0d:21:b1:db:2c:4d:ae:55:c9:14:77:5c 
The key's randomart image is:
+--[RSA1 2048]----+
|          o.+.ooE|
|           o O oo|
|          . = =  |
|           @ .   |
|        S = O    |
|         . =     |
|          .      |
|                 |
|                 |
+-----------------+
\end{verbatim}

Una vez que se han generado la llaves y el sistema a arrancado se inicia una
sesión como administrador del sistema:

\begin{verbatim}
 XUPV2P-MicheAngelo login: root
[root@XUPV2P-MicheAngelo /]#
\end{verbatim}

Como adminustrador del sistema se verifican los recursos disponibles:

\begin{verbatim}
[root@XUPV2P-MicheAngelo ~]# uname -a
Linux XUPV2P-MicheAngelo 3.2.0 
#6 Wed Oct 17 16:16:35 CDT 2012 ppc GNU/Linux
[root@XUPV2P-MicheAngelo ~]# cat /proc/cpuinfo 
processor       : 0
cpu             : Virtex-II Pro
clock           : 300.000000MHz
revision        : 8.160 (pvr 2001 08a0)
bogoppc        : 600.00
timebase        : 300000000
platform        : Xilinx Virtex
model           : testing
Memory          : 256 MB
[root@XUPV2P-MicheAngelo ~]# cat /proc/meminfo 
MemTotal:         252820 kB
MemFree:          242268 kB
Buffers:             240 kB
Cached:             6376 kB
SwapCached:            0 kB
Active:             4308 kB
Inactive:           3116 kB
Active(anon):        824 kB
Inactive(anon):        8 kB
Active(file):       3484 kB
Inactive(file):     3108 kB
Unevictable:           0 kB
Mlocked:               0 kB
HighTotal:             0 kB
HighFree:              0 kB
LowTotal:         252820 kB
LowFree:          242268 kB
SwapTotal:             0 kB
SwapFree:              0 kB
Dirty:                16 kB
Writeback:             0 kB
AnonPages:           816 kB
Mapped:             2012 kB
Shmem:                24 kB
Slab:               3584 kB
SReclaimable:        780 kB
SUnreclaim:         2804 kB
KernelStack:         216 kB
PageTables:          128 kB
NFS_Unstable:          0 kB
Bounce:                0 kB
WritebackTmp:          0 kB
CommitLimit:       62408 kB
Committed_AS:       2264 kB
VmallocTotal:     890880 kB
VmallocUsed:         160 kB
VmallocChunk:     890684 kB
[root@XUPV2P-MicheAngelo ~]#   
\end{verbatim}

\section{Compilación cruzada para Virtex-II Pro}

Gracias a \emph{buildroot} se cuenta ya con un entorno de compilación cruzada
capaz de generar archivos binarios que pueden ejecutarse desde la  tarjeta de
desarrollo de forma nativa, para poder compilar una aplicación para el entorno
generado es necesario configurar un par de variables de entorno :

\begin{lstlisting}[caption=Variables de entorno para compilación cruzada',]
export TOOLCHAIN_PPC=/opt/toolchains-ppc/buildroot-2011.05
export PATH=$TOOLCHAIN_PPC/output/host/usr/bin:$PATH
 
export AR=$TOOLCHAIN_PPC/output/host/usr/bin/ppc-linux-ar
export AS=$TOOLCHAIN_PPC/output/host/usr/bin/ppc-linux-as
export CC=$TOOLCHAIN_PPC/output/host/usr/bin/ppc-linux-gcc
export CPP=$TOOLCHAIN_PPC/output/host/usr/bin/ppc-linux-cpp
export CXX=$TOOLCHAIN_PPC/output/host/usr/bin/ppc-linux-g++
export LD=$TOOLCHAIN_PPC/output/host/usr/bin/ppc-linux-ld
export GCC=$TOOLCHAIN_PPC/output/host/usr/bin/ppc-linux-gcc
export NM=$TOOLCHAIN_PPC/output/host/usr/bin/ppc-linux-nm
\end{lstlisting}

Se procede a escribir una aplicación para probar la compilación cruzada:
\lstinputlisting[caption= Holavirtex.c,language=C]{./code/Holavirtex.c}

Ahora es  posible realizar la compilación cruzada desde el sistema huesped (x86)
hasta el sistema objetivo \emph{PPC045}, a través del binario
\emph{ppc-linux-gcc}, además vamos a agregar el \emph{flag static} para crear un
binario estático.
\begin{verbatim}
 ppc-linux-gcc /tmp/helloworld.c -static -s -o /tmp/helloworld-ppc
\end{verbatim}

El ejecutable generado debe de ser copiado y ejecutado en la tarjeta de
desarrollo como se indica :

\begin{verbatim}
 Proyecto@debian$ scp /tmp/helloworld-ppc root@192.168.1.1:/
\end{verbatim}

Y desde \emph{minicom} se ejecuta la aplicación:

\begin{verbatim}
 [root@XUPV2P-MicheAngelo ~]#  ./helloworld-ppc
 Hola Virtex!
 Hola Virtex!
 Hola Virtex!
 Hola Virtex!
 Hola Virtex!
 [root@XUPV2P-MicheAngelo ~]#
\end{verbatim}



