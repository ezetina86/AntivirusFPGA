\prefacesection{Resumen}

El desarrollo de aplicaciones de c\'omputo sobre dispositivos
empotrados, como lo son los FPGAs, tiene gran inter\'es para la industria de
la seguridad inform\'atica, ya que permite crear sistemas de c\'omputo
orientados a una tarea espec\'ifica sin la necesidad de un sistema de
c\'omputo completo.

Actualmente, existen multitud de sistemas de seguridad basados en
\emph{software}. Algunos proporcionan estabilidad aceptable y están integrados
en el sistema operativo, otros por ejemplo, son únicamente interfaces gráficas
que facilitan la tarea de gestionar la seguridad de un sistema de cómputo a la
mayoría de usuarios y, además, proporcionan seguridad por defecto para un uso
básico. Estos últimos, en entornos de mayor exigencia, no proporcionan una
respuesta fiable a un ataque informático.

Hoy en d\'ia los FPGAs son dispositivos que permiten la implementaci\'on de
complejos sistemas de seguridad basados en \emph{hardware} o \emph{software}
ofreciendo a los desarrolladores una alta gama de posibilidades de implementar
sus sistemas en ellos.

