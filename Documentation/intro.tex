\chapter{Introducción}

Con el incremento en el uso de internet han aumentado también las amenazas a los
sistemas de cómputo. Hoy en día, un sistema informático debe ser capaz de
realizar el filtrado de paquetes maliciosos en el tráfico de red. Sin embargo,
las herramientas de seguridad y aplicaciones de usuarios comparten un mismo
CPU\footnote{CPU, acrónimo de \emph{Central Processing Unit} se refiere a la
unidad central de proceso de una computadora.} con recursos para el
funcionamiento de herramientas de seguridad basadas
en \emph{software}.


Un sistema de cómputo se vuelve inseguro al encenderlo. Los huecos de
seguridad en un sistema de cómputo se manifiestan generalmente en
dos maneras:

\begin{itemize}
 \item Huecos de seguridad físicos, el problema sucede cuando se da
acceso  físico a una computadora a personas sin autorización, lo cual
le permitiría  realizar cosas que normalmente no sería capaz de hacer.
 \item Huecos de Seguridad en el \emph{Software}, donde los problemas son
causados por  puntos escritos incorrectamente en el \emph{software}, los cuales
pueden ser  utilizados para hacer cosas indebidas.
\end{itemize}

En la actualidad para evitar el acceso remoto de personas no
autorizadas a los sistemas  de c\'omputo es necesario robustecer más
los sistemas, lo que tiene repercución directa en el tiempo de
desarrollo de estos, además de hacerlos mas grandes y complejos.

Los sistemas embebidos son dispositivos de propósito específico que se utilizan
ampliamente en equipamientos de redes de datos, sensado remoto, comunicaciones,
etc. Estos equipos han sido desarrollados para llevar a cabo un conjunto de
tareas reducidas y específicas en función de su ámbito de desempeño,
característica que los diferencia de las computadoras de propósito general. El
sistema operativo embebido es quien le permite realizar sus funciones y en la
mayoría de los casos es provisto y desarrollado por el fabricante del
\emph{hardware.}

% Implementar un sistema embebido basado en un sistema operativo GNU/Linux que
% funcione de manera eficiente, robusta y sobretodo que permita modificaciones
% según necesidades puntuales es una tarea ardua y compleja.

Para resolver dichos problemas, este trabajo propone construir un sistema de
seguridad autónomo capaz de manipular y configurar reglas de seguridad en un
sistema operativo basado en \emph{Linux}, lo cual será de gran utilidad para la
administración y seguridad de una red de computadoras, ya que permitiría 
 detectar actividades mal intencionadas en el uso de la red.


\section{Motivación}

La información es hoy la materia prima de las organizaciones. Tener información
ayuda a tomar decisiones con seguridad y rapidez. Por tanto, proteger la
información en todo momento y permitir el acceso a ella sólo para las personas
que la necesiten y que, además, sea fiable, es un tema fundamental.

Los virus inform\'aticos son hoy una realidad reconocida por las empresas,
quienes saben que es un problema que mina su productividad, ya que sus
computadoras est\'an constantemente expuestas a vulnerabilidades de sus sistemas
de seguridad. El \emph{spam} afecta tambi\'en a las organizaciones, pues
eliminar todo el \emph{spam} que se recibe toma tiempo y, no s\'olo eso sino que
al ser excesivo provoca lentitud en los sistemas, lo que tiene una repercusi\'on
directa en la productividad.

El proyecto realizado describe el diseño de un sistema de seguridad de redes de
computadoras empotrado\footnote{Sistema Empotrado, traducci\'on del
ingl\'es \emph{embedded system}, es un sistema dise\~nado utilizando
componentes \emph{hardware} y \emph{software} en un \'unico m\'odulo y para una
aplicaci\'on espec\'ifica. Se pretende as\'i conseguir altas prestaciones y gran
flexibilidad con unos costos relativamente bajos y un consumo de
energ\'ia moderado.} en un FPGA\footnote{FPGA, acrónimo de \emph{Field
Programmable Gate Array}, es un dispositivo semiconductor que contiene bloques
de lógica cuya interconexión y funcionalidad pueden ser configuradas mediante un
lenguaje de descripción especializado.} con un sistema operativo \emph{Linux}
instalado, por lo que es  un sistema basado en \emph{hardware} y
\emph{software}.

El sistema es capaz de proteger la integridad, confiabilidad
y disponibilidad de los datos que ingresan o salen de una red de computadoras, 
ejecutando aplicaciones y servicios de seguridad a través de un sistema
operativo basado en \emph{Linux}. La red de computadoras que proteger\'a este
sistema puede no estar basada \emph{Linux}, lo que dar\'a al sistema la
posibilidad de proteger redes de computadoras con otros sistemas
operativos.


\section{Objetivos}

\subsection{Objetivo general}


% Diseñar e implementar un sistema de seguridad para redes de \'area local
% en la tarjeta de desarrollo NetFPGA\footnote{NetFPGA, es una plataforma de bajo
% costo, diseñada principalmente como una herramienta para la enseñanza de
% hardware de redes \cite{netfpga}.} que permita bloquear el tráfico no autorizado
% a una red y admita al mismo tiempo comunicaciones autorizadas.

Diseñar e implementar un sistema de seguridad para redes de \'area local
en la tarjeta de desarrollo \emph{XUPV2P} que permita bloquear el tráfico no
autorizado a una red y admita al mismo tiempo comunicaciones autorizadas.

\subsection{Objetivos específicos}
\begin{itemize} 

    \item Implementar un sistema operativo basado en Linux\footnote{Linux,
    es un núcleo libre usado en sistemas operativos de código
    abierto\cite{linux}.} en la tarjeta de desarrollo \emph{XUPV2P} el cual
    incluya los m\'odulos necesarios para la detecci\'on	
    de virus, filtrado de paquetes y acceso remoto.

    \item Construir en la tarjeta de desarrollo \emph{XUPV2P} un sistema capaz
    de detectar\emph{malware}\footnote{\emph{Malware}, es un tipo de software
    que tiene como objetivo infiltrarse o dañar una computadora sin el
    consentimiento de su propietario.} como troyanos\footnote{Troyano, es un
    software malicioso que se presenta al usuario como un programa aparentemente
    legítimo e inofensivo pero que al ejecutarlo ocasiona daños.},
    virus\footnote{Virus informático, es un programa con intenciones malignas,
    que es capaz de propagarse de una computadora a otra.} y otras amenazas
    maliciosas, con la finalidad de  proteger los datos que circulan por
    la red de amenazas o p\'erdidas.

    \item Configurar un \emph { firewall }\footnote{ \emph {Firewall} o
    cortafuegos, es un dispositivo configurado para permitir, limitar, cifrar y
     decifrar el tráfico de paquetes en una red de
    computadoras\cite{firewall}.} en la tarjeta de desarrollo \emph{XUPV2P},
    para filtrar paquetes y evitar intrusiones en la red.

    \item Configurar el servicio DHCP\footnote{DHCP, acrónimo de \emph{Dynamic
    Host Configuration Protocol}, es un protocolo que permite a dispositivos
    individuales en una red de direcciones IP obtener su propia información de
    configuración de red\cite{dhcp}.} en  en la tarjeta de desarrollo
    \emph{XUPV2P}, que permita  contar con una  lista de direcciones IP
    din\'amicas y asignarlas a los equipos conforme \'estas estan disponibles.

    \item Configurar en la terjeta de desarrollo \emph{XUPV2P} el servicio
    \emph{OpenSSH}\footnote{OpenSSH, acrónimo de \emph{Open Secure Shell}, es un
    conjunto de aplicaciones que permiten realizar comunicaciones cifradas a
    través de una red, usando el protocolo SSH\cite{openssh}.}, para poder
    acceder a el dispositivo de forma remota.

\end{itemize}
