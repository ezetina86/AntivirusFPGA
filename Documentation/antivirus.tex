\chapter{Virus}

\section{Introducción}

Un virus, es un \emph{software} o un fragmento de \emph{software} introducido
subrepticiamente en la memoria de una computadora que, al activarse, destruye
total o parcialmente la información almacenada, esta diseñado para propagarse de
un equipo a otro e interferir en el funcionamiendo de los mismos\cite{lehtinen}.

Los virus comienzan a ejecutarse cuando inicia el programa que lo contiene. El
virus puede reproducirse asi mismo, en mucho casos los virus necesitan
reproducirse para poder ser una verdadera amenaza. Los virus pueden
reporducirse de forma inmediata o hasta que algún evento que propicie su
reproducción. Por ejemplo, en una fecha exacta ( como con el famoso virus
\emph{Friday the 13th\footnote{\emph{Friday the 13th}, fue un virus
que consistía en borrar cada programa que fuera ejecutado en una
computadora, cada dia viernes 13.}})

Recientemente  fue necesario que un usuario actuvará un virus ejecutando un
programa corrupto, el cual pudo ser abierto desde algun archivo oculto en un
correo electrónico. La mayoria de los lectores de correo electrónico actuales
tienen la posibilidad de incluir archivos adjuntos dentro del correo
electrónico, por lo que atacantes buscan incluir \emph{malware} en estos
archivos.

La información es hoy la materia prima de las organizaciones. Tener información
ayuda a tomar decisiones con seguridad y rapidez. Por tanto, proteger la
información en todo momento y permitir el acceso a ella sólo para las personas
que la necesiten y que, además, sea fiable, es un tema fundamental.

Los virus inform\'aticos son hoy una realidad reconocida por las empresas,
quienes saben que es un problema que mina su productividad, ya que sus
computadoras est\'an constantemente expuestas a vulnerabilidades de sus sistemas
de seguridad.

Actualmente los FPGAs son dispositivos que permiten la implementaci\'on de
complejos sistemas de seguridad basados en \emph{hardware} o \emph{software}
ofreciendo a los desarrolladores una alta gama de posibilidades de implementar
sus sistemas en ellos.

\section{¿Qué es el  \emph{malware}?}

\emph{Malware}, es un \emph{software} o un fragmento de \emph{software}
diseñado para causar daños a los sistemas de cómputo, la expresión que viene de
la union de las palabras \emph{malicious} y \emph{software}, es un termino
general que cubre diferentes tipos de \emph{software} dañino\cite{avoine}.

El malware destructivo utiliza herramientas de comunicación conocidas para 
distribuir ``gusanos'' que se envían por correo electrónico y mensajes 
instantáneos, virus Troyanos que provienen de ciertos sitios Web y archivos 
infectados de virus que se descargan de conexiones P2P\footnote{Peer-to-Peer, 
red de pares, red entre iguales, red entre pares o red punto a punto (P2P, por 
sus siglas en inglés) es una red de computadoras en la que todos o algunos 
aspectos funcionan sin clientes ni servidores fijos, sino una serie de nodos que 
se comportan como iguales entre sí. Es decir, actúan simultáneamente como 
clientes y servidores respecto a los demás nodos de la red. Las redes P2P 
permiten el intercambio directo de información, en cualquier formato, entre los 
ordenadores interconectados.}. El malware también buscará explotar en silencio 
las vulnerabilidades existentes en sistemas.


\subsection{Tipos de malware}

\begin{itemize}

\item Virus. Es un programa que al ejecutarse, se propaga infectando a otros 
programas en la misma computadora.

\item Gusanos de Internet (\emph{worms}). Un gusano de Internet es un programa 
que se transmite a sí mismo, explotando vulnerabilidades en una red y así 
infectar otras computadoras.

\item Caballos de Troya (troyanos). Un troyano es un programa disfrazado como 
algo atractivo o inofensivo que invitan al usuario a ejecutarlo.

\item Puertas traseras (\emph{backdoors}). Una puerta trasera permite evadir 
los procedimientos normales de autenticación al conectarse a una computadora. 
Mediante un virus, un gusano de Internet o un troyano, se puede instalar una 
puerta trasera y así permitir un acceso remoto más fácil en el futuro.

\item \emph{Keyloggers}. Un keylogger es un programa que monitorea todo lo que 
el usuario teclea y lo almacena para un posterior envío. Por ejemplo, un número 
de tarjeta de crédito puede ser enviado al autor del programa y hacer pagos 
fraudulentos. La mayoría de los keyloggers son usados para recopilar claves de 
acceso y otra información sensible.

\item \emph{Botnets}. Las botnets son redes de computadoras controladas por un 
individuo con el fin de hacer envío masivo de spam o para lanzar ataques contra 
organizaciones afectando su ancho de banda impidiendo su correcto funcionamiento 
y usarlo como forma de extorsión.

\item \emph{Spyware}. Spyware es un programa que se instala en tu computadora, 
usualmente con el propósito de recopilar y luego enviar información a un 
individuo. 

\item \emph{Adware}. Son programas que muestran publicidad forma intrusiva e 
inesperada, usualmente en forma de ventanas emergentes (pop-up).

\item \emph{Ransomware}. También llamados secuestradores, son programas que 
cifran archivos importantes para el usuario, haciéndolos inaccesibles y así 
extorsionar al usuario para poder recibir la contraseña que le permita recuperar 
sus archivos.

\end{itemize}

\section{Antivirus}

Actualmente, existen multitud de sistemas de seguridad basados en
\emph{software}. Algunos proporcionan estabilidad aceptable y están integrados
en el sistema operativo, otros por ejemplo, son únicamente interfaces gráficas
que facilitan la tarea de gestionar la seguridad de un sistema de cómputo a la
mayoría de usuarios y, además, proporcionan seguridad por defecto para un uso
básico. Estos últimos, en entornos de mayor exigencia, no proporcionan una
respuesta fiable a un ataque informático.


\subsection{ClamAV}

ClamAV es una herramienta antivirus de código abierto (GPL) para sistemas UNIX.
ClamAV provee multiples  utilidades incluyendo  demonios felexibles y
escalables a  arquitecturas multinucleo, scanner a  travez de  lineas de
comandos y una  avanzada herramienta para actualizaciones automaticas.

\subsubsection{Caracteristicas}

\begin{itemize}
 \item Licencia  GNU (\emph{General Public License}, Version 2)
 \item Estandar de llamadas al sistema POSIX (\emph{Portable Operating System
Interface})
 \item Rapido escaneo
 \item Detección de mas de un millón de  virus, gusanos y troyanos incluyendo
virus de Macro Microdoft Office, \emph{malware} en dispositivos moviles y otras
amenazas.
\item Soporte para  escneo de  archivos  comprimidos incliyendo:
 \begin{itemize}
  \item Zip
  \item RAR
  \item 7Zip
  \item ARJ
  \item Tar
  \item CPIO
  \item Gzip
  \item Bzip2
 \end{itemize}
 
 \item Soporte para  escneo de  archivos portables ejecutables en plataformas de
32 y 64 bits incliyendo:
\begin{itemize}
 \item AsPack
 \item UPX
 \item PSG
 \item Petite
 \item wwpack32
\end{itemize}

\item  Soporte para archivos  ELF (32 y 64 bits)

\item  Soporta todos los  formatos de correo electronico

\item Soporte especial para formatos especiales  incluidos:
\begin{itemize}
 \item HTML
 \item RTF
 \item PDF
\end{itemize}
\item Actualizador de base de datos avanzado  con soporte para 
actualizaciones a travez de scripts, firmas digitales  y DNS.

\end{itemize}

\subsubsection{Plataformas Soportadas}

\subsubsection*{UNIX}

A partir de la  verisión 0.9 ClamAV soporta:
\begin{itemize}
 \item GNU/Linux
 \item Solaris
 \item FreeBSD
 \item OpenBSD
 \item Mac OS X
\end{itemize}

\subsubsection*{Windows}

Desde la  version 0.9  ClamAV tiene  una version nativa desarrollada bajo
Visual Stuido.


