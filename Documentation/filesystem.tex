\chapter{\emph{Root File System}}


\section{Introducción}

El sistema de archivos \emph{root file system} se encuentra localizado en la
misma partición en donde esta el directoio \emph{root} y es el sistema de
archivos en donde cualquier otro sistema de archivos serán montados.

Una partición es una seción lógica independiente de una unidad de
almacenamiento, un sistema de archivos es una jerarquia de directorios
utilizada para organizar los archivos de un istema de cómputo, en Linux, se
comienza con el directoio \emph{root}, el cual contiene una serie de
subdirectorios,cada uno  a su vez contiene directorios adicionales, etc. 


El contenido exacto del \emph{root file system}  puede variar en cada sistema
de cómputo, pero puede inclur los archivos necesarios para el arranque del
sistema, asi como también los archivos necesarios para montar otros sistemas de
archivos. El contenido del directorio \emph{root} junto con una cantidad mínima
de directorios incluyen los directorios \emph{/boot, /dev, /etc, /bin, /sbin}.

El  \emph{root file system} es generalmente pequeño, lo que ayuda a evitar que
el sistema de archivos sea corrupto, Un sistema de archivos corrupto puede
evitar el arranque del sistema\cite{rootfs}.

\section{\emph{BuildRoot}}

\emph{BuildRoot} es una serie de archivos de configuración que permiten la
creacion de un entorno de compilación cruzada mediante el uso de una
\emph{toolchain}, \emph{BuildRoot} es capaz de construir un sitema de archivos
(\emph{root file system}) o una imagen de Linux, ambas funciones pueden ser
usadas de forma independiente.

Además \emph{BuildRoot} proveé una inferstructura que permite reproducir el
proceso de desarrollo de un \emph{root file system}, esto es particularmente
útil cuando es necesario depurar, actualizar o agregar parches a un sistema de
archivos creado anteriormente\cite{buildroot}.

\emph{BuildRoot} se pude obtener suando Git\footnote{Git, es un software de
control de versiones diseñado por Linus Torvalds, pensando en la eficiencia y la
confiabilidad del mantenimiento de versiones de aplicaciones cuando estas tienen
un gran número de archivos de código fuente.} a travéz del siguiente comando :
\begin{verbatim}
  Proyecto@debian: git clone git://git.buildroot.net/buildroot
\end{verbatim}

\subsection{Configuración y uso general}

\emph{BuildRoot} tiene un configuración similar a la del \emph{kernel} de
Linux, por lo que se puede generar la configuración desde un asistente
ejecutando:
\begin{verbatim}
  Proyecto@debian: make menuconfig
\end{verbatim}

El archivo de configuración ``.config'' de \emph{BuildRoot}  se muestra en el
\emph{Apéndice B}.

Una vez que se han configurado todos los componentes que tendrá el \emph{root
file system} se procede a compilar el sistema de archivos con el comando:
\begin{verbatim}
  Proyecto@debian: make 
\end{verbatim}

Este comando realizará los siguientes pasos:
\begin{itemize}
 \item Descargara las fuentes necesarias.
 \item Configurará, instalará o importará la \emph{toolchain} que usara para
compilar los paquetes seleccionados en la configuración.
 \item Constriurá e instalará los paquetes objetivo.
 \item Creará el sistema de archivos en el formato seleccionado (\emph{ext2}
para este protecto).
\end{itemize}

La salida generada por \emph{BuildRoot} será almacenada en el directorio
\emph{``output/''}, este directorio contiene a su vez los siguientes
directorios:
\begin{itemize}
 \item \emph{``images/''}, en donde se encuentran todas las las imagenes 
guardadas
(kernel, sistema de archivos).
 \item \emph{``build/''}, contiene todos los compenentes que serán construidos.
 \item \emph{``staging/''}, contiene una jerarquía similar  ala que se generará
 el \emph{root file system}.
 \item \emph{``target/''}, contiene el \emph{root file system} completo.
 \item \emph{``host/''}, contiene las herramientas necesarias para la
compilación del \emph{root file system} desde el sistema huesped.
 \item \emph{``toolchain/''}, contiene los componentes de la \emph{toolchain}.
\end{itemize}



